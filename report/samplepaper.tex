	% This is samplepaper.tex, a sample chapter demonstrating the
% LLNCS macro package for Springer Computer Science proceedings;
% Version 2.20 of 2017/10/04
%
\documentclass[runningheads]{llncs}
%
\usepackage{graphicx}
% Used for displaying a sample figure. If possible, figure files should
% be included in EPS format.
%
% If you use the hyperref package, please uncomment the following line
% to display URLs in blue roman font according to Springer's eBook style:
% \renewcommand\UrlFont{\color{blue}\rmfamily}

\begin{document}
%
\title{Making History Not Count:\\
Should Historical Corpora Really Be Treated Differently for Event Detection Tasks?}
\titlerunning{Making History Not Count}

%
\titlerunning{Making History Not Count}
% If the paper title is too long for the running head, you can set
% an abbreviated paper title here
%
\author{Andrea Ferretti}
%
\authorrunning{A. Ferretti}
% First names are abbreviated in the running head.
% If there are more than two authors, 'et al.' is used.
%
\institute{University of Milan, Milan, Italy \\
\email{andrea.ferretti1@studenti.unimi.it}}
%
\maketitle              % typeset the header of the contribution
%
\begin{abstract}
The abstract should briefly summarize the contents of the paper in
150--250 words.

\keywords{Event Detection  \and Historical Event Detection \and Glove \and \\ Embeddings comparison.}
\end{abstract}
%
%
%
\section{Introduction}
%\subsection{A Subsection Sample}
Word embeddings aim to map words of a vocabulary to vectors of numbers. This is done in order to have a more versatile, tractable, and mathematical representation of those words and improve the performances of several natural language processing tasks. The more intuitive way to do so would be to have a one-hot encoding representation of each word in the vocabulary. This method, however doesn't provide any information about the meaning of a word. To allow the vectors to retain semantic meaning the foundamental idea of distributional semantics is used: words have similar meaning if they appear in similar contexts. This translates to, given a corpus of documents, representing a word as a distribution, over the vocabulary, of the frequency of the words that appear in its context in the corpus. These vectors have the limit of being highly dimensional and extrimely sparse: every word would be represented by a vector of tens of thousands dimensions (the size of the vocabulary) the vast majority of which would have value zero.

To solve these problems several techniques either based on neural networks or matrix factorization have been proposed \cite{embeddings}. They both start from the sparse and high dimensional co-occurence matrix and obtain a fixed length, dense, and real valued vector for each word. The vectors are not interpretable when taken singularly, but when analyzed and compared to one another they show interesting properties: words that have similar meaning (in the distributional semantics sense) have vectors that are closed to each other according to the cosine or Euclidean distance, and pairs of vectors representing pairs of words analogy, such as man and woman, king and queen, also have the similar distances. Word2Vec \cite{wtv} and GloVe \cite{glove} are two embeddings algorithms, with comparable performances, based respectively on neural networks and matrix factorization techniques.

Event detection is one of the various tasks that makes use of word embeddings.


\subsubsection{Sample Heading (Third Level)} Only two levels of
headings should be numbered. Lower level headings remain unnumbered;
they are formatted as run-in headings.

\paragraph{Sample Heading (Fourth Level)}
The contribution should contain no more than four levels of
headings. Table~\ref{tab1} gives a summary of all heading levels.

\begin{table}
\caption{Table captions should be placed above the
tables.}\label{tab1}
\begin{tabular}{|l|l|l|}
\hline
Heading level &  Example & Font size and style\\
\hline
Title (centered) &  {\Large\bfseries Lecture Notes} & 14 point, bold\\
1st-level heading &  {\large\bfseries 1 Introduction} & 12 point, bold\\
2nd-level heading & {\bfseries 2.1 Printing Area} & 10 point, bold\\
3rd-level heading & {\bfseries Run-in Heading in Bold.} Text follows & 10 point, bold\\
4th-level heading & {\itshape Lowest Level Heading.} Text follows & 10 point, italic\\
\hline
\end{tabular}
\end{table}


\noindent Displayed equations are centered and set on a separate
line.
\begin{equation}
x + y = z
\end{equation}
Please try to avoid rasterized images for line-art diagrams and
schemas. Whenever possible, use vector graphics instead (see
Fig.~\ref{fig1}).

\begin{figure}
\includegraphics[width=\textwidth]{fig1.eps}
\caption{A figure caption is always placed below the illustration.
Please note that short captions are centered, while long ones are
justified by the macro package automatically.} \label{fig1}
\end{figure}

\begin{theorem}
This is a sample theorem. The run-in heading is set in bold, while
the following text appears in italics. Definitions, lemmas,
propositions, and corollaries are styled the same way.
\end{theorem}
%
% the environments 'definition', 'lemma', 'proposition', 'corollary',
% 'remark', and 'example' are defined in the LLNCS documentclass as well.
%
\begin{proof}
Proofs, examples, and remarks have the initial word in italics,
while the following text appears in normal font.
\end{proof}
For citations of references, we prefer the use of square brackets
and consecutive numbers. Citations using labels or the author/year
convention are also acceptable. The following bibliography provides
a sample reference list with entries for journal
articles~\cite{ref_article1}, an LNCS chapter~\cite{ref_lncs1}, a
book~\cite{ref_book1}, proceedings without editors~\cite{ref_proc1},
and a homepage~\cite{ref_url1}. Multiple citations are grouped
\cite{ref_article1,ref_lncs1,ref_book1},
\cite{ref_article1,ref_book1,ref_proc1,ref_url1}.
%
% ---- Bibliography ----
%
% BibTeX users should specify bibliography style 'splncs04'.
% References will then be sorted and formatted in the correct style.
%
% \bibliographystyle{splncs04}
% \bibliography{mybibliography}
%
\bibliography{biblio}
\bibliographystyle{ieeetr}
%
%\bibitem{ref_article1}
%Author, F.: Article title. Journal \textbf{2}(5), 99--110 (2016)

%\bibitem{ref_lncs1}
%Author, F., Author, S.: Title of a proceedings paper. In: Editor,
%F., Editor, S. (eds.) CONFERENCE 2016, LNCS, vol. 9999, pp. 1--13.
%Springer, Heidelberg (2016). \doi{10.10007/1234567890}

%\bibitem{ref_book1}
%Author, F., Author, S., Author, T.: Book title. 2nd edn. Publisher,
%Location (1999)

%\bibitem{ref_proc1}
%Author, A.-B.: Contribution title. In: 9th International Proceedings
%on Proceedings, pp. 1--2. Publisher, Location (2010)

%\bibitem{ref_url1}
%LNCS Homepage, \url{http://www.springer.com/lncs}. Last accessed 4
%Oct 2017

\end{document}
